\documentclass[a4paper]{article}

\usepackage[backend=bibtex]{biblatex}
\addbibresource{./sources.bib}
\usepackage{hyperref}

\begin{document}

\section*{Announcement: Seminar on characteristic classes}

This is the announcement for the upcoming (online) seminar on 
\textbf{characteristic classes of manifolds and singular spaces}
organized by 
\href{mailto:njunior@icmc.usp.br}{Nivaldo Grulha}, 
\href{mailto:zach@math.uni-hannover.de}{Matthias Zach} and 
\href{mailto:essig@math.uni-kiel.de}{Timo Essig}.

\begin{center}
	\begin{tabular}{rl}
		Time & Tuesday, 2pm Brazilian time (= 7pm European summer time) \\
			&  \href{https://uni-kiel.zoom.us/j/69103354611?pwd=TUlmeGZQNlVPSG5YcEtDdndDU1kydz09}{Zoom link} \\
		Zoom info	& Meeting ID: 691 0335 4611 Passcode: 539478
	\end{tabular}
\end{center}

\subsubsection*{Idea of the seminar} 
Since the participants are scattered all over the world, 
we plan to have it online. 
This is a research oriented seminar and some kind of mix of a classical seminar,
a research discussion and a reading course.
One session should consist of a concise presentation of the 
day's topic, lasting approximately half an hour, and then mostly well prepared and 
guided examples and exercises. Thus, it is the speaker's responsibility to gather 
and organize the material in such a way that the other participants obtain the 
possibility to learn the topic ``hands on''. 

\subsubsection*{Organization of the seminar}
Information will be distributed via the 
\href{https://github.com/timoessig/CCSingSem/wiki}{Github Wiki}
created for the seminar. Files like the one you are looking at are saved on that
repository as well. If you have any questions concerning the wiki or git in general,
feel free to email Timo.

It would be nice if everyone who is considering participation could send an 
email to Matthias, in which they briefly describe 
their particular topics of interest, further suggestions and, if applicable, 
connections to their own research. 
%I will then try to extract a coherent 
%outline with concrete talks and topics from the gathered material. 
%Also, feel free to approach me via email or BBB/Skype/Jitsi/Zoom if 
%you have any questions or would like to discuss these matters. 

\subsection*{Overview on the topics of the seminar}
Disclaimer: To not get lost into too many technical details, we focus on the complex setting
and mainly work with Chern classes.

\begin{enumerate}
\item \textbf{Introduction} by \emph{M. Zach, T. Essig} on 03/29/2022 \\
	Examples for applications of characteristic classes on smooth manifolds are the
	starting point for the seminar. We present some relations to fiber bundle theory 
	and to bordism theory. Via the Poincar\'e-Hopf-Theorem we get to Grothendieck's 
	conjecture for the existence of Chern classes on singular spaces.
	%
\item \textbf{CCs on manifolds via obstruction theory 1} by \emph{Edmundo Martins} on 04/05/2022
	%
\item \textbf{CCs on manifolds via obstruction theory 2} by \emph{Edmundo Martins} on 04/12/2022\\
	In these two talks we learn about the obstruction theory apporach - an algebraic 
	topology approach to characteristic classes.
	%
\item \textbf{CCs on manifolds via differential forms} by \emph{Ivan Tezoto} on 04/19/2022 \\
	An alternative approach to Chern classes on smooth manifolds following Bott-Tu (?)
	%
\item \textbf{Functorial approach to CCs on manifolds} by \emph{M. Zach} on 04/26/2022 \\
	This talk contains a more axiomatic approach to CCs on manifolds, the axioms
	for them and the question for a general framework that covers the different approaches.
	We examine the Riemann-Roch-Theorem in its different forms.
\end{enumerate}
The topics for the other talks are not fully determined yet. After these first talks, we want
to focus on characteristic classes on singular spaces. We cover MacPherson's solution to 
the Grothendieck conjecture first and understand his construction of Chern classes using the
local Euler obstruction (\cite{MacPherson74,Parusinski06}). After that, we examine the Fulton 
class and the Milnor class, that measures the deviation of the two aforementioned classes.

\subsection*{Unordered collection of ideas and topics}
In the following, you can find a rather unordered list of topics that will be or might be
covered in the seminar. We will pick the interesting and reachable topics during the seminar,
so we do not chain ourselves to the list.
\subsubsection*{Characteristic classes of manifolds% (using vector bundles and other approaches)
}
\begin{itemize}
  \item Introduce the Chern class calculus: What are the axioms. How does it work? 
    Why is it useful? 
  \item Present some explicit approaches to Chern classes: Chern-Weyl theory for 
    smooth de Rham cohomology, algebraic Chern classes taking values in the 
    duals of Chow groups, Chern classes via topological obstruction theory, 
    and Chern classes of holomorphic vector bundles taking values in algebraic 
    de Rham cohomology.

    Can we see examples for these? Can we explain in which sense these theories 
    are ``all the same'', i.e. can we describe the functoriality? 
%
  \item How can other characteristic classes, such as Pontrjagin and $L$-classes be defined?
	Present the alternative approach to $L$-classes via maps to spheres and explain again,
	why theses theories are ``the same''.
  \item What is bordism and surgery, why are these concepts important and what is the relation
	to characteristic classes?
%
  \item $K$-theory: What is the $K$-group $K^0(X)$ of a topological space $X$? 
    What is the Grothendieck group of coherent sheaves $K_0(X)$? 
    What can it be used for? Spoiler: These $K$-groups should sit at the top 
    of all the theory of characteristic classes that were discussed in the 
    previous paragraph in the sense that all characteristic classes are functors 
    factoring through these $K$-groups. Can we make these functors explicit? 
  \item What is a Riemann-Roch-type theorem? 
  \item What does Motivic integration have to do with this? 
\end{itemize}

\subsubsection*{Characteristic classes for singular varieties}
\begin{itemize}
  \item What does the ``Grothendieck-Deligne conjecture'' say? Explain this 
    in explicit examples. 
  \item Explain the original work of MacPherson \cite{MacPherson74}: 
    What is the local Euler obstruction and what is it used for? 
  \item What is a ``characteristic class'' in general? There are others, 
    such as, for instance, the $L$-class. 
    How is the $L$-class defined for sing. varieties in contrast to manifolds?
\end{itemize}

\subsubsection*{Characteristic classes in singularity theory}
\begin{itemize}
  \item What is the ``Buchsbaum-Rim multiplicity''? 
  \item What are the Fulton-Johnson and the Milnor classes?
  \item What is the Euler obstruction of a $1$-form? 
\end{itemize}

\subsection*{References} 
\subsubsection*{Characteristic classes on manifolds}
\begin{itemize}
  \item \cite{MilnorStasheff74}: Rather broad book on characteristic classes of real and 
    complex vector bundles.
  \item \cite{Husemoeller94}: Touches upon obstruction theory. 
  \item \cite{BottTu82}: Treats, for instance, Characteristic classes with values in smooth Cech-de-Rham 
    cohomology in an elementary way.
  \item \cite{Fulton98}: The reference for Chern classes and intersection theory 
    of algebraic schemes; mostly using Chow groups.
  \item \cite{Hirzebruch66}: Monograph on the proof of the Riemann-Roch-Theorem.
  \item \cite{Stong68}: Short (and steep) introduction to CCs in Chapter 5.
  \item \cite{Atiyah56}: Covers the case of complex analytic vector bundles on 
    projective manifolds using Hodge theory.
  \item The notes by R. McEnroe (can be found in the folder on GitHub) give a good overview on
	  Milnor's construction of exotic spheres and the role that characteristic classes have.
\end{itemize}

\subsubsection*{Characteristic classes on singular spaces}
The first four references are survey like. All of them have different goals and come from 
different time frames.
\begin{itemize}
  \item \cite{Yokura05} is a survey that is "an appendix to MacPherson's" survey, which
	  I had a hard time finding in the internet. A newer survey is \cite{Yokura07}.
  \item \cite{SchuermannYokura07} is a rather general survey with a nice exposition,
	  which later focusses on motivic CCs. Those are probably beyond the scope of
	  our seminar.
  \item \cite{Brasselet00} surveys CCs on manifolds via obstruction theory and Schubert
	  cycles and the generalizations of these ideas to singular spaces. It is a good
	  overview and does not cover many technical details.
  \item \cite{Parusinski06} is a survey on Milnor classes, which also covers the basics on
	  the MacPherson and Fulton classes. The approach is via characteristic cycles.
  \item \cite{MacPherson74} is MacPhersons original article on his approach to the Chern
	  class for singular spaces. He makes use of the local Euler obstruction in the
	  construction. We will focus on this article in the second part of the seminar.
  \item \cite{BaumFultonMacPherson79} and \cite{BaumFultonMacPherson75}: 
    discuss the Fulton class and Riemann-Roch-type theorems for both algebraic 
    and topological $K$-theories.
%  \item \cite{Banagl07} Book on intersection homology, that includes L-classes for manifolds 
%	and singular spaces and explain, why they are important wrt surgery theory. Tends to
%	be a bit technical, but in particular the part about the motivation in Chapter 5 is good.
%  \item \cite{Maxim19} Another book on intersection homology, focussing on the sheaf theoretic approach.
%	Also covers $L$-classes, but in a bit more streamlined (less detailed and exact way). I recommend it.
\end{itemize}


\printbibliography

\end{document}
