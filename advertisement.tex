\documentclass[a4paper]{article}

\usepackage[backend=bibtex]{biblatex}
\addbibresource{sources.bib}

\usepackage{amsmath}
\usepackage{amssymb}
\newcommand{\D}{\mathrm{d}}

\begin{document}

\section{Characteristic classes on singular spaces -- continued}

After our introductory talks during the past semester, we would like to 
continue our seminar on Characteristic classes on singular varieties. 
The tentative schedule is the following.

\subsection{Part 1: Characteristic classes for singular varieties}
To start the semester, we want to understand MacPherson's solution of the
Grothendieck-Deligne conjecture. In particular, we want to get a better 
feeling of the local Euler obstruction and MacPherson's graph conjecture.
The talks are more or less organized as follows.
\begin{enumerate}
\item \emph{Preliminaries, Nash-blowup, Mather class, local Euler obstruction}


    Explain the ``Grothendieck-Deligne conjecture'' and compare it to char. 
    classes on manifolds. Introduce the Nash blowup and Nash bundle.
    Define the local Euler obstruction. In the best case,
    show a good example of a sing. variety embedded in some smooth variety.
    (E.g. Whitney umbrella) (Mostly 1. -- 4. of \cite{MacPherson74}).
    Note, that there alternative, algebraic approachs to the local 
    Euler obstruction by Gonzalez-Sprinberg, \cite{Gonzalez81}, 
    which is (partly) translated to English by Jiang, \cite{Jiang19}.
    See also: \cite{Kennedy90}.
\item Explain the graph construction and how it helps to prove the MacPherson-Chern 
    class theorem. (Chapters 5. -- 7. of MacPherson).
\end{enumerate}

To this end, we propose to start with a reading of \cite{MacPherson74} 
with an emphasis on constructing \textit{explicit examples} for the theory 
developed. 

\subsection{Part 2: Riemann-Roch type theorems in singularity theory}
Consider the classical Riemann-Roch theorem for a line bundle $\mathcal L$ 
on a smooth projective curve $C$:
\begin{equation}
  h^0(C, \mathcal L) - h^0(C, \mathcal L^\vee \otimes \omega_C) 
  = 
  \deg(\mathcal L) + 1 - g.
  \label{eqn:RRforCurves}
\end{equation}
Using Serre duality, one can rewrite the left hand side as 
\[
  h^0(C, \mathcal L) - h^0(C, \mathcal L^\vee \otimes \omega_C) 
  = h^0(C, \mathcal L) - h^1(C, \mathcal L) = \chi(\mathcal L)
\]
the \textit{holomorphic Euler characteristic} of $\mathcal L$. 
On the right hand side, we find a purely \textit{topological}
invariant given by the degree
\[
  \deg(\mathcal L) = \int_C c_1(\mathcal L) = \int_C \mathrm{Eu}(\mathcal L),
\]
which is nothing but the integral over the Euler class, and 
a correction term given by the \textit{genus} $g$ of the curve. 
In this spirit, whenever we speak of a ``Riemann-Roch type theorem'', 
we mean a theorem that does exactly this: Express a topological 
invariant as a holomorphic Euler characteristic. 

\medskip

\textbf{Where does this occur in Singularity theory?} -- 
One example is Milnor's formula for an isolated hypersurface singularity
$f\colon (\mathbb C^{n+1}, 0) \to (\mathbb C, 0)$: 
\[
  \mu(f) = \dim_\mathbb C \mathcal O_{n+1}/\mathrm{Jac}(f).
\]
The left hand side is the topological obstruction to extending the 
$1$-form $\D f$ from a small sphere $S_\varepsilon^{2n+1}\subset \mathbb C^{n+1}$
around the origin as a nowhere vanishing section to its interior. 
The right hand side can be interpreted as the holomorphic Euler characteristic 
of the \textit{Koszul-complex} $(\Omega^*_{\mathbb C^{n+1}, 0}, \D f \wedge - )$ 
induced by the differential of $f$. 

\medskip

\textbf{Why is this interesting?} -- Topological invariants are notoriously 
hard to compute. For algebraic invariants, such as holomorphic Euler characteristics, 
we do have a chance using computer algebra systems. Thus, whenever we can express 
a topological invariant in terms of the latter, we have a chance to actually get 
our hands on it. As a tentative goal for this seminar I propose that we reach a 
position in which we can effectively solve the following problem:

\begin{centering}
  ``Given $f \in \mathbb{Q}[x_0,\dots,x_n]$, compute a Whitney stratification for 
  the variety $X = V(f)$ and the constructible function $\mathrm Eu$ given by 
  the local Euler obstruction along the strata. ''
\end{centering}

As a starting point for the reading we propose \cite{BaumFultonMacPherson79} 
to first look into the following questions:
\begin{itemize}
  \item What is the $K$-group $K^0(X)$ of a topological space $X$? 
  \item What is the Grothendieck group of coherent sheaves $K^{\mathrm{alg}}_0(X)$? 
    How is it covariant? How does this relate to the derived category of $X$?
  \item What is its topological analogue $K^{\mathrm{top}}_0(X)$? 
  \item How are the Chern characters defined in either theory? Why do the 
    obvious diagrams commute? 
  \item How do the classical Riemann-Roch theorems follow from this? 
\end{itemize}

After this, we turn towards the local theory. Gonzalez-Sprinberg has 
given an algebraic formula for the local Euler obstruction via an involved 
blowup procedure \cite{Gonzalez81} that we could start with. 
I propose to go through the proof and see how things can be rephrased so that 
one could possibly obtain another algebraic formula without the blowups -- 
possibly through some projection formula. 
(Note, that there is a (partial) English translation of the Gonzales-Sprinberg 
paper by Jiang in \cite{Jiang19}.)

As a tentative direction, I would like to go towards algebraic residues
as described in \cite[Chapter 5]{GriffithsHarris78}. 
To explain this, recall from \cite{GriffithsHarris78} that the multidimensional 
residue of a rational $n+1$-form 
\[
  \omega = \frac{h(z)\D z_0 \wedge \dots \wedge \D z_n}{f_0 \cdots f_n}
\]
for some regular sequence $f_0,\dots,f_n \in \mathcal{O}_{n+1}$ 
can be transformed into a sphere integral:
\[
  \frac{1}{(2\pi \sqrt{-1})^{n+1}}
  \int_{|f_0| = r_0} \cdots \int_{|f_n| = r_n} \omega
  =
  \int_{S_\varepsilon^{2n+1}} \eta_\omega
\]
where $\eta_\omega$ is the \textit{dolbeault representative} of $\omega$.
While the left hand side can be evaluated algebraically, at least 
for algebraic input, the right hand side has the form of a local 
topological obstruction: It could arise as the degree of a map of 
spheres. In fact, this leads to a residue formula for the Milnor 
number, see \cite{BrasseletLehmannSeadeSuwa99}. 

In the case where $(X,0) \subset (\mathbb{C}^{n+1}, 0)$ is the affine cone 
of a projective variety $Y \subset \mathbb{P}^n$, we can in fact 
compute the local Euler obstruction of $(X,0)$ at the origin 
via an algebraic residue calculus. 
The hope is to find 
an ``easy'' algebraic formula for the local Euler obstruction 
which extends this result in general. 


\subsection{References} 
\begin{itemize}
  \item \cite{MilnorStasheff74}: Rather broad book on characteristic classes of real and 
    complex vector bundles.
  \item \cite{Husemoeller94}: Touches upon obstruction theory. 
  \item \cite{BottTu82}: Treats, for instance, Characteristic classes with values in smooth Cech-de-Rham 
    cohomology in an elementary way.
  \item \cite{Fulton98}: The reference for Chern classes and intersection theory 
    of algebraic schemes; mostly using Chow groups.
  \item \cite{BaumFultonMacPherson79} and \cite{BaumFultonMacPherson75}: 
    Discusses Riemann-Roch-type theorems for both algebraic 
    and topological $K$-theories.
  \item \cite{Atiyah56}: Covers the case of complex analytic vector bundles on 
    projective manifolds using Hodge theory.
  \item \cite{Banagl07} Book on intersection homology, that includes L-classes for manifolds 
	and singular spaces and explain, why they are important wrt surgery theory. Tends to
	be a bit technical, but in particular the part about the motivation in Chapter 5 is good.
  \item \cite{Maxim19} Another book on intersection homology, focussing on the sheaf theoretic approach.
	Also covers $L$-classes, but in a bit more streamlined (less detailed and exact way). I recommend it.
\end{itemize}

\printbibliography

\end{document}
